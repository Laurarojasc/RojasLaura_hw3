%--------------------------------------------------------------------
%--------------------------------------------------------------------
% Formato para los talleres del curso de Métodos Computacionales
% Universidad de los Andes
%--------------------------------------------------------------------
%--------------------------------------------------------------------

\documentclass[11pt,letterpaper]{exam}
\usepackage[utf8]{inputenc}
\usepackage[spanish]{babel}
\usepackage{graphicx}
\usepackage{tabularx}
\usepackage[absolute]{textpos} % Para poner una imagen en posiciones arbitrarias
\usepackage{multirow}
\usepackage{float}
\usepackage{hyperref}
%\decimalpoint

\begin{document}
\begin{center}
{\Large Métodos Computacionales} \\
Resultados de tarea 3 Métodos \\ 
Laura Rojas Cardenas\\
\end{center}


\noindent
\section{Gr\'aficas ejercicio 2: Trayectoria de una particula en cargada en un campo magenetico dado }
\begin{center}
\includegraphics[width=10cm]{xcontray.png} 
\end{center}
Se observa una grafica circular porque solo hay fuerzas en x y y 

\begin{center}
\includegraphics[width=10cm]{xcontraz.png} 
\end{center}
En esta grafica se observa un comportamiento en forma de espiral

\begin{center}
\includegraphics[width=10cm]{ycontrat.png} 
\end{center}
En esta grafica se logra observar una curva de forma trigonometrica 

\begin{center}
\includegraphics[width=10cm]{posicion.png} 
\end{center}
Grafica similar a de x contra z  



\section{Gr\'aficas ejercicio 3: Ecuacion de onda en dos dimensiones para una membrana cuadrada}

\begin{center}
\includegraphics[width=10cm]{Condiciones_iniciales.png} 
\end{center}
En esta grafica se observan las condiciones iniciales de la posicion, en donde se logra apreciar un pico maximo. 


\begin{center}
\includegraphics[width=10cm]{membrana_fijos_60ms.png} 
\end{center}
La grafica muestra un pico al igual que la anterior, sin embargo este es un poco menos pronunciado debido a que son extremos fijos

\begin{center}
\includegraphics[width=10cm]{membrana_libres_60ms.png} 
\end{center}
La grafica muestra un pico al igual que la anterior, sin embargo este es un poco menos pronunciado debido a que son extremos libres


\begin{center}
\includegraphics[width=10cm]{membrana_corte_fijos.png} 
\end{center}
dldld 

\begin{center}
\includegraphics[width=10cm]{membrana_corte_libres.png} 
\end{center}
dldld 





\end{document}

